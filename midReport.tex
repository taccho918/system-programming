\documentclass[9pt]{jarticle}
\usepackage[top=30truemm,bottom=30truemm,left=25truemm,right=25truemm]{geometry}
\begin{document}
\begin{center}
\begin{large}
システムプログラミングレポート \\
演習課題:  \\
\end{large}
09426532 高橋 桃花 \\
出題日: 2015年4月13日 \\
提出日: 2015年7月13日 \\
締切日: 2015年7月13日 \\
\end{center}

%%%%%%%%%%%%%%%%%%%%%%%%%%%%%%%
\section{概要}
%%%%%%%%%%%%%%%%%%%%%%%%%%%%%%%


\subsection{課題1-1}
教科書A.8節「入力と出力」に示されている方法と,A.9節最後「システムコール」に示されている方法のそれぞれで"Hello \ World"を表示せよ.両者の方式を比較し考察せよ.

\subsection{課題1-2}
アセンブリ言語中で使用する.data, .text及び.alignとは何か解説せよ.下記コード中の6行目の.dataがない場合,どうなるかについて考察せよ.

\subsection{課題1-3}
教科書A.6節「手続き呼出し規約」に従って,関数factを実装せよ.(以降の課題においては,この規約に全て従うこと)factをC言語で記述した場合は,以下のようになるであろう.

\subsection{課題1-4}
素数を最初から100番目まで求めて表示するMIPSのアセンブリ言語プログラムを作成してテストせよ.その際,素数を求めるために下記の2つのルーチンを作成すること.

\subsection{課題1-5}
素数を最初から100番目まで求めて表示するMIPSのアセンブリ言語プログラムを作成してテストせよ.ただし,配列に実行結果を保存するようにmain部分を改造し,ユーザの入力によって任意の番目の配列要素を表示可能にせよ.

%%%%%%%%%%%%%%%%%%%%%%%%%%%%%
\section{プログラムの作成方針}
%%%%%%%%%%%%%%%%%%%%%%%%%%%%%






%%%%%%%%%%%%%%%%%%%%%%%%%%%%%
\section{重視したこと}
%%%%%%%%%%%%%%%%%%%%%%%%%%%%%







%%%%%%%%%%%%%%%%%%%%%%%%%%%%%
\section{プログラムリスト}
%%%%%%%%%%%%%%%%%%%%%%%%%%%%%

\subsection{課題1-1}
\begin{verbatim}
	.text
	.align 2
	
main:
	move	$s0, $ra		# mainを呼んだ戻り先のアドレスが入っている
							# $raを$s0に保存しておく
	
	li	$a0, 72			# 'H' = 72, $a0 = H 
	jal	putc			# jamp and link to putc
	li	$a0, 101		# 'a' = 101, $a0 = e
	jal	putc			# 
	li	$a0, 108		# 'l' = 108, $a0 = l
	jal	putc			#
	li	$a0, 108		# 'l' = 108, $a0 = l
	jal	putc			# 
	li	$a0, 111		# 'o' = 111, $a0 = o
	jal	putc			# 
	la	$a0, 32
	jal	putc			# print " "
	li	$a0, 87			# 'W' = 87, $a0 = W
	jal	putc			#
	li	$a0, 111		# 'o' = 111, $a0 = o
	jal	putc			#
	li	$a0, 114		# 'r' = 114, $a0 = r
	jal	putc			#
	li	$a0, 108		# 'l' = 108, $a0 = l
	jal	putc			#
	li	$a0, 100		# 'd' = 100, $a0 = d
	jal	putc			#

	move	$ra, $s0		# $s0に保存しておいた戻り先を$raに入れる
	j	$ra					# mainを呼んだ戻り先に飛ぶ
putc:
	lw	$t0, 0xffff0008		# $t0 = *(oxffff0008)
	li	$t1, 1				# $t1 = 1
	and 	$t0, $t0, $t1	# $t0 =& $t1
	beqz	$t0, putc		# if( $t0==c ) goto putc
	sw	$a0, 0xffff000c		# *(0xffff000c) = $a0
	j	$ra

\end{verbatim}

\subsection{課題1-2}
\begin{verbatim}
	.data
	.align 2
str:
	.asciiz "Hello World"		# "Hello World"の文字列を定義

	.text
	.align 2
main:								
	li	$v0, 4					# print_stringのシステム・コール・コード	
	la	$a0, str 				# プリントする文字列
	syscall						# strを出力
	
\end{verbatim}

\subsection{課題1-3}




%%%%%%%%%%%%%%%%%%%%%%%%%%%%%
\section{プログラムの使用法}
%%%%%%%%%%%%%%%%%%%%%%%%%%%%%





%%%%%%%%%%%%%%%%%%%%%%%%%%%%%
\section{プログラムの作成過程に関する考察}
%%%%%%%%%%%%%%%%%%%%%%%%%%%%%




%%%%%%%%%%%%%%%%%%%%%%%%%%%%%
\section{得られた結果に関する考察,あるいは設問に対する回答}
%%%%%%%%%%%%%%%%%%%%%%%%%%%%%

\subsection{課題1-1}
.data:\ 文字列をプログラムのデータ・セグメントに格納する.データ・セグメントとは,ソールファイル中のデータの2進数表現を保持するオブジェクトファイルの1セクションである.

.text: \ 命令をテキスト・セグメントに格納する.テキスト・セグメントとは,ソースファイル中のルーチンに対応する機械語コードを保持する.

.align: \ 

\end{document}

























